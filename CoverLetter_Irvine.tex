\documentclass[11pt,letter]{article}
\usepackage[utf8]{inputenc}
\usepackage[english]{babel}
\usepackage{fancyhdr}
\usepackage{lastpage}
\usepackage[dvipsnames]{xcolor}
\usepackage[lastpage,user]{zref}
\usepackage{graphicx}

\color{darkgray}
\definecolor{royalblue}{HTML}{4183D7}

\usepackage[colorlinks = true,
linkcolor = royalblue,
urlcolor  = royalblue,
citecolor = royalblue,
anchorcolor = royalblue,
breaklinks = true]{hyperref}

\setlength\parindent{0pt}

\usepackage[left=2cm,top=4cm,right=2cm,bottom=2cm,headsep=2cm]{geometry}

\RequirePackage{fancyhdr}
\fancyhf{}
\chead{
  {\Large\scshape\color[HTML]{CF000F} Giordon Stark}\\
  University of Chicago\\
  5640 South Ellis Avenue, PRC~161\\
  Chicago, IL\ \ \ 60637
}
\lhead{\footnotesize\bfseries
  \href{mailto:kratsg@uchicago.edu}{kratsg@uchicago.edu}
}
\rhead{\footnotesize\bfseries
  \href{https://giordonstark.com}{giordonstark.com}
}
\fancyheadoffset{-1cm}

\begin{document}

\thispagestyle{fancy}

Professor Anyes Taffard \\
University of California, Irvine \\
3166 Frederick Reines Hall \\
Irvine, CA 92697 \\[0.5cm]

Dear Profs. Lankford, Taffard, Whiteson, \\
\\
My name is Giordon Stark and my experience with the ATLAS collaboration make me an ideal candidate for the high energy physics post-doc position (\href{https://recruit.ap.uci.edu/apply/JPF04404}{JPF04404}) at University of California, Irvine. I am a Ph.D. candidate in Physics at the University of Chicago, planning to graduate this Spring, June 2018. My PI is Professor David W. Miller. Your group has a strong physics program in searches for supersymmetry and lepton-flavour-violating decays of the Higgs boson, as well as being involved in instrumentation upgrades of the ATLAS detector, such as the Phase I muon upgrade with the New Small Wheel and the Phase II muon trigger upgrade using the Monitoring Drift Chambers. I know that with my background in supersymmetry searches and instrumentation upgrades of TDAQ in the ATLAS detector, I am a good fit for the UCI group.\\
\\
For the duration of the Run 2 program of the LHC, I have been involved in an analysis search for supersymmetry with hadronic final states. This analysis has two publications so far and known as setting one of the strongest limits on gluino masses in both the ATLAS and CMS collaborations. I have been involved in every aspect of this analysis and it is a very strong team effort that is able to produce results. Additionally, my work on the instrumentation upgrades often comes with a lack of manpower, so I find myself working on my own projects and pioneering new concepts, such as the use of an embedded processor or building a custom operating system. I am equally comfortable working as a member of a team as well as independently, but I also know when to ask for help.\\
\\
I am also passionate about education and outreach. I have been involved in numerous ATLAS Induction Days and put together material for many of the tutorials on a wide variety of topics from using analysis code, to understanding the trigger system. I ensure that all the projects I work on also have adequate documentation so that others can use it as well. I have also filmed the first few videos in American Sign Language with CERN as part of the Microcosm exhibit to improve accessibility and diversity. I am also involved in my local community here in Chicago to improve physical accessibility to theater and broadway. \\
\\
Please find enclosed my curriculum vitae and research statement. There are also three reference letters submitted by: Professor David Miller (\href{mailto:david.w.miller@uchicago.edu}{\nolinkurl{david.w.miller@uchicago.edu}}), Dr. Zachary Marshall (\href{mailto:zach.marshall@cern.ch}{\nolinkurl{zach.marshall@cern.ch}}), and Dr. Michael Begel (\href{mailto:begel@cern.ch}{\nolinkurl{begel@cern.ch}}). Thank you for considering my application. I look forward to hearing from you soon. Should you have any questions, you may reach me using the contact information listed above. \\

Sincerely, \\
\includegraphics[height=3\baselineskip]{signature}\\
\textbf{Giordon Stark}

\end{document}
